\begin{titlepage}
   \begin{center}
       \vspace*{1cm}
       \huge
       \textbf{Time and frequency domain analysis of a low-pass $RC$-filter}\\
       \vspace{1.5cm}
        \large
       Gabriel Mazili Pedroza -- gmp40, lab group 47\\
       \vspace{0.5cm}
       11 November 2024\\
       \vspace{0.5 cm}
       \textit{Homerton College, University of Cambridge}
       \vspace{1.5 cm}
        \begin{abstract}
            As part of the coursework for the Integrated Electrical Project, the features of a low-pass $RC$-filter were dimensioned with an oscilloscope. The aims were to compare the measured characteristic time $\tau$ and attenuation curve (including its roll-off) to the theoretical predictions, to quantify the impact of component tolerances in gathered data, and to test if $\tau$ values chosen within said tolerances could improve the agreement with theory. The attenuation data matched theory without theoretical adjustments, but changes to the idealised circuit model were required to enhance agreement in the transient response experiment. Furthermore, choosing an \textit{ad hoc} value for $\tau$ improved the agreement in both cases, although its closeness to the nominal value meant the impact of tolerances on circuit behaviour were largely unobserved in experiments. 
        \end{abstract}
       \vfill
            
   \end{center}
\end{titlepage}